\section{Introduction}
\subsection{Purpose}
Agriculture plays a pivotal role in India’s economy as over 58\% of rural households depend on it as the principal means of livelihood, 80\% of whom are smallholder farmers with less than 5 acres of farmland. More than a fifth of the smallholder farm households are below poverty. \\
The COVID-19 pandemic has greatly highlighted the massive disruption caused in food supply chains exposing the vulnerabilities of marginalized communities, small holder farmers and the importance of building resilient food systems. It has become even more important now that we develop and adopt innovative methodologies and technologies that can help bolster countries against food supply shocks and challenges.\\
The purpose of Data-dRiven prEdictive fArMing (DREAM) is to help Telangana’s government in designing, developing, and demonstrating anticipatory models for food systems using digital public goods and the community of agricultural worker.
To achieve this DREAM allows the communication between the users, gives the possibility to retrieve data from the sensors in Telangana, to track the evolutions of the work of farmers, and to access the data from farmers or from any other sources.

\subsubsection{Goals}
The ambition of this project is that the adoption of DREAM will help the sustainability of the agricultural workers in Telangana. This is crucial during a pandemic emergency, like the one that is going on, because the farmers are the ones more afflicted. Below are presented the goals of DREAM. 
\begin{enumerate}[label=\textbf{G.\arabic*}]
	\item Make available the weather forecasts collected by Telangana.

	\item Create an anticipatory model for Telangana’s food system.

    \item Allow the communication between 2 users.

	\item Support the agricultural work in Telangana.
	\begin{enumerate}[label=\textbf{G.4.\arabic*}]
	    \item Support the work of the farmers.
	    \item Support the work of the agronomists.
	    \item Support the work of the TPM.
	\end{enumerate}

\end{enumerate}

\subsection{Scope}
Data-dRiven prEdictive fArMing (DREAM) is an easy-to-use and intuitive application which aims to settle for various problems faced by the member of the agricultural community of Telangana.\\

It allows Telangana's Policy Makers (TPM) to access data and analyze the performance of all farmers in the country, identify which farmers are coping well with adversity and which ones need help, understanding which initiatives are performing best, in order to replicate them in the whole Nation, and they can contact the resilient farmers to give them special incentives.\\
To improve their cultivations, farmers can get personalized suggestions, report any problem they encountered during their work, request more support by an agronomist or another farmer, discuss in the about any topic in the forum, but they can also provide their data of production.\\
Agronomists can use this system to increase the effectiveness of their work as they can manage their daily work and keep track of the visits they have planned throughout the year and evaluating the performance of farmers they are responsible of. Furthermore, they have access to the weather forecast and the data collected by the system to better take decisions with farmers or receive requests for help from farmers.

\subsubsection{Shared Phenomena}
\textbf{Farmer}
\begin{enumerate}[label=\textbf{SP.\arabic*}]
    \item Farmer logs into DREAM
    \item Farmer chats with another user
    \item Farmer reports a problem he/she faced
    \item Farmer asks for help
    \item Farmer replies to a request of help
    \item Farmer interacts in the Forum
    \item Farmer gets suggestions
    \item Farmer inserts data of production
    \item Farmer searches the weather forecast
    \item Resilient farmer shares information on how he/her overcame a meteorological adverse event\newline
    \\
\textbf{Agronomist}
    \item Agronomist logs into DREAM
    \item Agronomist chats with another user
    \item Agronomist reads about a problem faced by a farmer
    \item Agronomist replies to a request of help
    \item Agronomist gets suggestions
    \item Agronomist looks for data of production of a farmer
    \item Agronomist analyzes the performances ranking of farmers
    \item Agronomist searches the weather forecast
    \item Agronomist visualizes his/her daily plan
    \item Agronomist updates his/her agenda
    \item Agronomist uploads his/her report about a visit to a farmed he/she has just concluded\newline
    \\
 \textbf{TPM}
    \item TPM logs into DREAM 
    \item TPM chats with another user
    \item TPM reads about a problem faced by a farmer
    \item TPM looks for data of production of a farmer
    \item TPM analyzes the performances ranking of farmers
    \item TPM searches for the initiatives taken to help a farmer
    \item TPM ask to a Resilient Farmer to provide best practices on how he/she overcame a meteorological adverse event\newline
    \\
\textbf{DREAM}
    \item System shows the weather forecast for the time slot selected
    \item System shows the ranking of the farmers
    \item System shows the suggestions requested by a farmer of an agronomist
    \item System shows the daily plan for the selected day
    \item System shows the production data of the selected user
    \item System shows the initiatives taken to help a farmer
\end{enumerate}

\subsubsection{World Phenomena}
\begin{enumerate}[label=\textbf{WP.\arabic*}]
    \item Farmer faces a problem
    \item Farmer faces a meteorological adverse event
    \item Farmer collects data of the production
    \item Agronomist visits a farmer
    \item Agronomist chooses which farms to visit
    \item Agronomist helps a farmer to improve his/her production
    \item TPM provide special incentives to resilient farmers
    \item Resilient Farmer receives incentives from Telangana
\end{enumerate}
\newpage
\subsection{Glossary}
This section explains the terms used throughout the document.

%TODO: Sistemare sta cosa delle tabelle
\subsubsection{Definitions}
\begin{center}
    \begin{tabular}{@{}p{0.35\linewidth} p{0.65\linewidth}}
        \hline
        \textbf{Term} & \textbf{Definition}\\
        \hline
        \textbf{Good Farmers} & Farmers who performed well\\
        \textbf{Resilient Farmers} & Farmers who performed well, despite they faced an adverse weather event\\
        \textbf{Bad Farmers} & Farmers that are not able to sustain their production\\
        \textbf{Farmer’s Performance} & It is the average quantity produced per acre by a farmer\\
        \textbf{Problem faced by farmers} & Any difficulty that a farmer faced during his/her work, i.e., a bug infestation of the cultures\\
        \textbf{Temporal Slot} & An arch of time selected by the user to see the weather. E.g., it could be just one day, but also 6 months\\
        \textbf{Type of Production} & How the cultivation is grown, i.e., Shifting Cultivation, intensive, and many more\\
        \textbf{Data of Production} & Information concerning a single cultivation of a farmer. It includes tons produced per acre of land used, fertilizer used, humidity of the soil, type of production, water usage, seed planted and the zone of the cultivation\\
        \textbf{Credentials} & Username and Password given by Telangana to a user to access DREAM\\
        \textbf{Short-Term Forecast} & Weather forecast concerning 7 days or less\\
        \textbf{Long-Term Forecast} & Weather forecast concerning more than 7 days\\
        \textbf{Zone} & It refers to a District and a Mandala\\
        \textbf{Meteorological Adverse Event} & A climate event that caused many damages to a farmer\\
        \hline
    \end{tabular}
\end{center}

\subsubsection{Acronyms}
\begin{center}
    \begin{tabular}{p{0.35\linewidth} p{0.65\linewidth}}
    \hline
        \textbf{Acronym} & \textbf{Term}\\
        \hline
        \textbf{DREAM} & Data-dRiven PrEdictive FArMing in Telangana\\
        \textbf{TPM} & Telangana Policy Makers\\
        \textbf{GUI} & Graphical User Interface\\
        \textbf{GPS} & Global Positioning System\\
    \hline
    \end{tabular}
\end{center}

\subsubsection{Abbreviations}
\begin{center}
    \begin{tabular}{p{0.35\linewidth} p{0.65\linewidth}}
    \hline
        \textbf{Abbreviation} & \textbf{Term}\\
    \hline
        \textbf{e.g.} & Exempli Gratia\\
        \textbf{i.e.} & Id est\\
        \textbf{G} & Goal\\
        \textbf{WP} & World Phenomena\\
        \textbf{SP} & Shared Phenomena\\
        \textbf{R} & Requirement\\
        \textbf{DA} & Domain Assumption\\
        \textbf{UC} & Use Case\\
    \hline
    \end{tabular}
\end{center}
\newpage

\subsection{Reference Documents}
Below is presented the list of all the documents used to create this document.
\begin{itemize}
    \item Project assignment specification document
    \item ISO/IEC/IEEE 29148 - Systems and software engineering
    \item Course slides on WeBeeP
    \item UNDP India’s \href{https://github.com/UNDP-India/Data4Policy}{Github}
    \item Telangana’s Forecast \href{https://www.tsdps.telangana.gov.in/aws.jsp}{WebPage}
    \item \href{https://www.agrifarming.in/agriculture-farming-in-telangana-and-schemes}{Agriculture farming} in Telangana
    \item Package to insert the Alloy code \href{https://github.com/Angtrim/alloy-latex-highlighting/blob/master/example.tex}{Github}
    
\end{itemize}

\subsection{Document Structure}
This document is presented as it follows:
\begin{enumerate}
    \item \textbf{Introduction} contains a preamble of the given problem and proposes, in a simple way, the system and its goals as solution
    \item \textbf{Overall Description} gives a general description of the system, focusing on its functions and constraints. Moreover, it provides the domain assumptions of the analyzed world
    \item \textbf{Specific Requirements} explains in detail the functional and non-functional requirements. It lists the possible interactions with the system in the form of scenarios, use cases and sequence diagrams
    \item \textbf{Formal Analysis} contains the Alloy model of some critical aspects of the system and an example of the generated world
    \item \textbf{Effort Spent} keeps track of the time spent to complete this document. The first table defines the hours spent as a team for taking the most important decisions or reviewing contents, the seconds contain the individual hours spent working on this project
\end{enumerate}